% Crucial Preamble
\documentclass[12pt,letterpaper]{article} \usepackage{amsmath} \usepackage{graphicx} \usepackage[margin=1in]{geometry} \usepackage{longtable}  \usepackage{amssymb}

% Extra Preamble
\usepackage{fancyhdr} \usepackage{enumitem} \usepackage{float} \usepackage{soul}
\usepackage{multicol} \usepackage[compact]{titlesec}


% frames with display breaks
\usepackage{mdframed}
\allowdisplaybreaks

% change spacing
\usepackage{setspace}
\setlength{\parskip}{0.4\baselineskip}

% Remove paragraph indentation
\setlength{\parindent}{0pt}

% Reduce space before and after section headings
%\titlespacing*{\section}{0pt}{0.1\baselineskip}{0.2\baselineskip}

% changes font
%\renewcommand{\familydefault}{\sfdefault}

% adds header and footer
\pagestyle{fancy}
\fancyhead{} \fancyhead[C]{MAT 2377 Cheat Sheet} \fancyhead[L]{MAT2377} \fancyhead[R]{Owen Daigle}
\fancyfoot{} \fancyfoot[C]{\thepage}


\begin{document}
	
	\begin{center}
		\Large\textbf{MAT 2377 Cheat Sheet} \\
		\vspace{0.5em}
	\end{center}
	
	\section{Chapter 1: Probabilities}
	\subsection{Sample Spaces}
	The \textbf{sample space} is the set of all possible outcomes. 
	
	An \textbf{event} is a collection of outcomes in the sample space. Usually this is what we are looking to work with. 
	
	\subsection{Counting Techniques}
	We can count items using the $k$ stage procedure. 
	
	If we have $k$ stages, each with $n_1$, $n_2$, $n_3$, ... possibilities, then the total number of possiblilites is just $n_1\cdot n_2\cdot n_3\cdot ...\cdot n_k$.
	
	\subsection{Ordered Samples}
	If we have an ordered sample, then we see that picking $1, 2, 3$ is different than picking in a different order $1, 3, 2$.
	
	If we draw r items from a bag of n items:
	\begin{itemize}[]
		\item If we replace each item after drawing, we have: $n\cdot n\cdot n\cdot ... = n^r$ possibilities
		\item If we do NOT replace the items, we have: $n\cdot (n-1)\cdot (n-2)\cdot ...\cdot (n-r) = \frac{n!}{(n-r)!}= {}_nP_r$
	\end{itemize}
	
	\subsection{Unordered Samples}
	
	
	\subsection{Conditional Probabilities}
	
	\subsection{Law of Total Probability}
	
	\subsection{Bayes Theorum}
	
	\section{Chapter 2: Discrete Random Variables}
	
\end{document}