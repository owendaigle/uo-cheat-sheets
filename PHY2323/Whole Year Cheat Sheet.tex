% Crucial Preamble
\documentclass[12pt,letterpaper]{article} \usepackage{amsmath} \usepackage{graphicx} \usepackage[margin=1in]{geometry} \usepackage{longtable}  \usepackage{amssymb}

% Extra Preamble
\usepackage{fancyhdr} \usepackage{enumitem} \usepackage{float} \usepackage{soul}
\usepackage{multicol} \usepackage[compact]{titlesec}


% frames with display breaks
\usepackage{mdframed}
\allowdisplaybreaks

% change spacing
\usepackage{setspace}
\setlength{\parskip}{0.4\baselineskip}

% Remove paragraph indentation
\setlength{\parindent}{0pt}

% Reduce space before and after section headings
%\titlespacing*{\section}{0pt}{0.1\baselineskip}{0.2\baselineskip}

% changes font
%\renewcommand{\familydefault}{\sfdefault}

% adds header and footer
\pagestyle{fancy}
\fancyhead{} \fancyhead[C]{PHY 2323 Cheat Sheet} \fancyhead[L]{PHY2323} \fancyhead[R]{Owen Daigle}
\fancyfoot{} \fancyfoot[C]{\thepage}


\begin{document}
	
	\begin{center}
		\Large\textbf{PHY 2323 Cheat Sheet} \\
		\vspace{0.5em}
	\end{center}
	
	\section{Coordinate Systems}
	There are 3 main coordinate systems:
	\begin{enumerate}[]
		\item Cartesian ($x,y,z$)
		\item Cylindrical ($\rho, \phi, z$)
		\item Spherical ($r,\phi, \theta$)
	\end{enumerate}
	
	\section{Electric Fields}
	
	\subsection{Coulombs Law}
	Coulomb's law is used to sum up all the charges in a location which will give an electric field $E$.
	\begin{align*}
		\vec E (\vec r) = \frac{1}{4\pi \epsilon_0} \int_S \frac{\rho(\vec r\prime)(\vec r - \vec r\prime)}{|\vec r - \vec r\prime | ^3}\mathrm d l
	\end{align*}
	This can also be extended to a surface with $\mathrm ds$ and 2 integrals, or volume with $\mathrm dv$
	
	Note that anything with the $\prime$ means that it is related to the \textbf{surface of charge}, and anything without the prime is related to the \textbf{observation point.}
	
	\subsection{Gausses Law}
	Gausses Law can be used on a \textbf{closed surface} where we make a guassian surface (such as a sphere, or cylander) at the point of interest. 
	\begin{align*}
		\int_S \vec E \mathrm d \vec s = \frac{Q_{enc}}{\epsilon}
	\end{align*}

	We also have the $\vec D$ field which is the \textit{Electric Flux Density}.
	\begin{align*}
		\int_S \vec D \mathrm d \vec s = Q_{enc}
	\end{align*}
	
	Finally, we have the flux $\psi$, a scalar. 
	\begin{align*}
		\psi = \epsilon \int_s \vec E \mathrm d \vec s = \int \vec D \mathrm d \vec s
	\end{align*}

	This is useful in 3 main cases. 
	\begin{enumerate}[]
		\item Spherical Symmetry is present
		\item Cylindrical symmetry is present (long line of charge with uniform $\rho$ or cylander with no angular dependance)
		\item Planar Symmetry (Long 2D surface of charge)
	\end{enumerate}

	A useful piece of information is if we want to find the $Q_{enc}$, we can often just integrate the charge density in a volume $V$.
	\begin{align*}
		Q_{enc} = \iiint_V \rho \mathrm d V
	\end{align*}

	Also, the $\vec D$ field is just the $\vec E$ field times a factor of $\epsilon$ ($\vec D = \epsilon \vec E$) 
	
	\section{Electric Potential}
	This is the potential energy per unit charge. AKA the voltage. This is a \textbf{scalar field}. This is \textit{independant of the path chosen}. 
	\begin{align*}
		V(\vec r) = \frac{1}{4\pi \epsilon _0} \int_{l\prime }\frac{\rho _l \mathrm d l\prime}{|\vec r-\vec r\prime| }
	\end{align*}
	This can be extended into 2d or 3d space by changing the $l\prime$ and $\mathrm d l\prime$ for $s\prime, \mathrm d s\prime$ or $v\prime, \mathrm d v\prime$.
	
	Then we can relate the change in voltage to the electric field:
	\begin{align*}
		\vec E = -\nabla V \qquad \nabla V = -\int \vec E\cdot \mathrm d \vec l
	\end{align*}

	\subsection{Electric Dipole}
	A \textbf{dipole} is a pair of equal and opposite charges that are very close to each other relative to the point of observation.
	
	This means that at the point of observation, they seem as one charge. 
	
	We have an equation that relates the charge of each end of the dipole $q$, the distance between the charges $d$, and the vector between the dipole and the observation point $\vec r$. This vector must be large compared to $d$.
	\begin{align*}
		V(\vec r) = \frac{(qd) \hat z \cdot \hat r}{4\pi \epsilon |r|^2}
	\end{align*}

	\subsection{Capacitors}
	The capacitence $C$ can be calculated using the following formula:
	\begin{align*}
		C=\frac{Q}{\Delta V}
	\end{align*}
	In practice, we use the following 3 step procedure:
	\begin{enumerate}[noitemsep]
		\item Find $E$
		\item Find $\Delta V$
		\item Find $C$ using $C=\frac{Q}{\Delta V} $
	\end{enumerate}
	
	\section{Materials in Electric Fields}
	There are 3 types of materials:
	\begin{enumerate}[]
		\item Conductors
		\item Insulators
		\item Semiconductors
	\end{enumerate}

	We have rules concerning the normal and tangential part of the boundary between 2 materials. 
	
	To obtain the normal part, we take the unit vector of the boundary, and this is our normal vector ($\hat n = \frac{\vec n}{|n|}$).
	
	If we then want to get the \textbf{normal part} of $E$, we dot product it with $\hat n$, then append $n$ to keep direction ($\vec E_n = (\vec E\cdot \hat n)\hat n$)
	
	The \textbf{tangential part }is just $\vec E_t = \vec E - \vec E_n$

	\subsection{Boundary between 2 Dielectrics}
	If we have 2 electric fields between 2 \textbf{dielectric} (insulators) surfaces, we have the following formulas for the bounds:
	\begin{align*}
		E_{1t} = E_{2t} \qquad \epsilon_1E_{1n}-\epsilon_2 E_{2n} = \rho_s
	\end{align*}

	\subsection{Surface of a Conductor}
	If we have an electric field at the \textbf{surface }of a \textbf{conductive} surface (note that inside the surface $E=0$), we have the following formulas for the bounds:
	\begin{align*}
		E_{1t} = E_{2t} = 0 \qquad E_{n} = \frac{\rho_s}{\epsilon_0}
	\end{align*}
	
	\section{Energy Stored in an Electric Field}
	We say that $W$ is the energy stored in an electric field, or the energy required to assemble a charge distribution. 
	
	
	
\end{document}