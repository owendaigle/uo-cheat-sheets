% Crucial Preamble
\documentclass[12pt,letterpaper]{article} \usepackage{amsmath} \usepackage{graphicx} \usepackage[margin=1in]{geometry} \usepackage{longtable}  \usepackage{amssymb}

% Extra Preamble
\usepackage{fancyhdr} \usepackage{enumitem} \usepackage{float} \usepackage{soul}
\usepackage{multicol} \usepackage[compact]{titlesec}


% frames with display breaks
\usepackage{mdframed}
\allowdisplaybreaks

% change spacing
\usepackage{setspace}
\setlength{\parskip}{0.4\baselineskip}

% Remove paragraph indentation
\setlength{\parindent}{0pt}

% Reduce space before and after section headings
%\titlespacing*{\section}{0pt}{0.1\baselineskip}{0.2\baselineskip}

% changes font
%\renewcommand{\familydefault}{\sfdefault}

% adds header and footer
\pagestyle{fancy}
\fancyhead{} \fancyhead[C]{PHY 2323 Cheat Sheet} \fancyhead[L]{PHY2323} \fancyhead[R]{Owen Daigle}
\fancyfoot{} \fancyfoot[C]{\thepage}


\begin{document}
	
	\begin{center}
		\Large\textbf{PHY 2323 Cheat Sheet} \\
		\vspace{0.5em}
	\end{center}

	\section{Chapter 1, 2: Math Review}
	
	\subsection{Coordinate Systems}
	
	\section{Chapter 3: Electrostatics}
	
	\subsection{Coulombs Law}
	\begin{align*}
		\vec E (\vec r) = \frac{1}{4\pi \epsilon_0} \int_S \frac{\rho(\vec r\prime)(\vec r - \vec r\prime)}{|\vec r - \vec r\prime | ^3}\mathrm d l
	\end{align*}
	This can also be extended to a surface with $\mathrm ds$ and 2 integrals, or volume with $\mathrm dv$
	
	\subsection{Gausses Law}
	Gausses Law can be used on a \textbf{closed surface} where we make a guassian surface (such as a sphere) at the point of interest. 
	\begin{align*}
		\int_S \vec E \mathrm d \vec s = \frac{Q_{enc}}{\epsilon}
	\end{align*}

	We also have the $\vec D$ field which is the \textit{Electric Flux Density}.
	\begin{align*}
		\int_S \vec D \mathrm d \vec s = Q_{enc}
	\end{align*}
	
	Finally, we have the flux $\psi$, a scalar. 
	\begin{align*}
		\psi = \epsilon \int_s \vec E \mathrm d \vec s = \int \vec D \mathrm d \vec s
	\end{align*}

	This is useful in 3 main cases. 
	\begin{enumerate}[]
		\item Spherical Symmetry is present
		\item Cylindrical symmetry is present (long line of charge with uniform $\rho$ or cylander with no angular dependance)
		\item Planar Symmetry (Long 2D surface of charge)
	\end{enumerate}
	
	\subsection{Electric Potential}
	This is the potential energy per unit charge. AKA the voltage. This is a \textbf{scalar field}. This is \textit{independant of the path chosen}. 
	\begin{align*}
		V(\vec r) = \frac{1}{4\pi \epsilon _0} \int_{l\prime }\frac{\rho _l \mathrm d l\prime}{|\vec r-\vec r\prime| }
	\end{align*}
	This can be extended into 2d or 3d space by changing the $l\prime$ and $\mathrm d l\prime$ for $s\prime, \mathrm d s\prime$ or $v\prime, \mathrm d v\prime$.
	
	Then we can say also at a higher level that:
	\begin{align*}
		\vec E = -\nabla V \qquad \nabla V = -\int \vec E\cdot \mathrm d \vec l
	\end{align*}

	\subsection{Electric Dipole}
	A \textbf{dipole} is a pair of equal and opposite charges that are very close to each other relative to the point of observation.
	
	This means that at the point of observation, they seem as one charge. 
	
	\subsection{Materials in Electric Fields}
	There are 3 types of materials:
	\begin{enumerate}[]
		\item Conductors
		\item Insulators
		\item Semiconductors
	\end{enumerate}

	If we have 2 electric fields between 2 dielectric (insulators) surfaces, we have the following formulas for the D and E fields. 
	
	
\end{document}