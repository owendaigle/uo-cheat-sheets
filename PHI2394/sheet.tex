% Crucial Preamble
\documentclass[12pt,letterpaper]{article} \usepackage{amsmath} \usepackage{graphicx} \usepackage[margin=1in]{geometry} \usepackage{longtable}  \usepackage{amssymb}

% Extra Preamble
\usepackage{fancyhdr} \usepackage{enumitem} \usepackage{float} \usepackage{soul}
\usepackage{multicol} \usepackage[compact]{titlesec}
\usepackage{listings}
\usepackage{pdfpages}



% frames with display breaks
\usepackage{mdframed}
\allowdisplaybreaks

% change spacing
\usepackage{setspace}
\setlength{\parskip}{0.4\baselineskip}

% Remove paragraph indentation
\setlength{\parindent}{0pt}

% Reduce space before and after section headings
%\titlespacing*{\section}{0pt}{0.1\baselineskip}{0.2\baselineskip}

% changes font
%\renewcommand{\familydefault}{\sfdefault}

% adds header and footer
\pagestyle{fancy}
\fancyhead{} \fancyhead[C]{PHI 2394 Summary Sheet} \fancyhead[L]{PHI 2394} \fancyhead[R]{Owen Daigle}
\fancyfoot{} \fancyfoot[C]{\thepage}

\definecolor{mygreen}{rgb}{0,0.6,0}
\definecolor{mygray}{rgb}{0.5,0.5,0.5}
\definecolor{mymauve}{rgb}{0.58,0,0.82}

\lstset{ %
	backgroundcolor=\color{white},   % choose the background color
	basicstyle=\small,        % size of fonts used for the code
	breaklines=true,                 % automatic line breaking only at whitespace
	captionpos=b,                    % sets the caption-position to bottom
	commentstyle=\color{mygreen},    % comment style
	escapeinside={<@}{@>},			 % Escapes these chars
	keywordstyle=\color{blue},       % keyword style
	stringstyle=\color{mymauve},     % string literal style
}


\begin{document}
	
	\begin{center}
		\Large\textbf{PHI 2394} \\
		\vspace{0.5em}
	\end{center}	
	
	\textbf{DISCLAIMER}: I have no idea what I am doing in this course, I am just guessing (educated guessing) really since I am not a philosophy student, not even close, and do not really understand the content.
	
	\tableofcontents
	\newpage
	
	\section{Summary}
	\subsection{Greeks}
	Greeks thing of technology as something that imitates nature. Weaving and house building were created by imitating spiders and swallows building nets and nests. 

	Technology is a genuine mechanical knowledge such as how farmers know how to work the land, and shoemakers know how to work with wood and other materials. This is known as Techne. Episteme is the theoretical knowledge about techne. 

	They organized science into multiple categories such as Logic, Science and Philosophy, Ethics and Politics, and Art and Technology.

	The Greeks thought of nature (physis) as a divine law, the process of coming to be. They thought that everything done according to nature (or with technology since that perfects nature) is divine. Technology can only seek to perfect nature by altering it to meet human needs and to be useful. 
	
	
	\subsection{What is Technology}
	When coming up with a definition the definition should not be too broad or narrow, it should not be circular, it should not use figurative language or metaphors, and it should be in positive terms. 
	
	Technology can be defined as tools or machines. These items are concrete, and easily understandable. This definition however does not include things such as physiologist tools, which are technology. 
	
	Technology can also be defined as a pattern of rule following behavior. This treats technology as rules rather than tools. Technology is a pattern that is systematically developed. This will include psychology, and software. 
	
	
	\subsection{Francis Bacon}

	Bacon is the founder of the modern inductive scientific method where we observe something, and come up with a rule to explain that something. This is done systematically. 
	
	The human mind has a lot of problems, it has 4 main biasses in which it sees the world. 
	\begin{itemize}
	\item Idol of the tribe means that we are biased due to human nature, we want to have order and regularity, and we are influenced by emotions. 
	\item Idol of the cave are illusions of the individual person, a single person thinks the whole world is similar to this person and thinks in the same way.
	\item Idol of the marketplace is about language barriers such as words that have complex or abstract meanings, and miscommunication. 
	\item Idols of the theater are about superstitions.
	\end{itemize}

	Bacon thought that humans should dominate and control nature. We however need to do this by obeying the laws of nature. Humans are superior to nature. 

The final goal of humans is to create a new reality. 

	Bacon wanted to reform the education system to take precaution against the idols.
	
	
	\subsection{August Compte}
	He introduced and popularized positive philosophy. It is about knowledge that is testable, and verified. 
	
	He said that there were three stages in humanities development. Theological stage (Sacred Text), Metaphysical stage (Theoretical Reason), and Scientific stage (Experimentation and Observation). The first stage is dominated by military, the second stage dominated by churches, and lawyers, and the third stage by industrial institutions and science. Individual human knowledge also follows these three stages. 
	
	Sociology is the end of the scientific (positive) method. All other sciences build up to it. 

	He believes that the fragmentation of science is not a good thing, they should not be split up into all the different stages. We should focus on one general specialty that studies the general scientific traits 
	Comte wanted to reform the education system with sociology.
	
	
	\subsection{Jean Jacque Rousseau}
	Rousseau valued freedom, he considers technology as removing freedom. Humans are increasingly dependent on one another for their satisfaction of their needs. 

	Humans are naturally good, a human in nature would be free and be good. Civilization corrupted the humans by forcing them to compete for things such as food, and water. Since humans have to fight other people for this in a society with technology, they will steal and kill. 

	Science adds luxury and leisure which do not improve the moral well being. Luxury is needed to feed the poor people, but if there was no luxury, then there would be no poor people. 
	
	
	\subsection{Jacques Ellul}
	Continental philosophy of technology tends to criticize technology, this is what Ellul studies. 

	There are 7 main characteristics of technology. Rationality, artificiality, automatism, self-augmentation, wholeness, universalism, and autonomy.

	Technology is meant to create efficiency. This can take away from the thing that makes people human. It is created for the modern world, but it is not a good fit for society. Technology has outgrown human control, we are not able to understand or govern all these individual technologies. 

	Technique is the prime mover of all the rest of the change such as social change, and political change. Technology is autonomous, it will keep marching forward even without humans. It is its own system with its own laws, outside of human values. 

	Ellul believes that the world is deterministic. Freedom is dynamic, and humans must continue to be determined to be free. 

	He believes that everything now is rooted in technique. Even sports use technology which reduces freedom, and gives deterministic results.

	He does not want to eliminate technology, but wants to raise awareness and make it known that individuals have a responsibility to be aware of technology’s flaws in that it is deterministic. 
	
	
	\subsection{Mario Bunge}
	There are two types of philosophy of technology. The rationalist view (Continental philosophy of technology) has a critical view of technology. The empiricism view (Analytic philosophy of technology) has a more positive view of technology. Mario Bunge has an analytic view, he views a very strong connection between technology and philosophy. He thinks this because of the three main pillars of philosophy which are metaphysical, ethical, and epistemological. 
	
	We can talk about technology in all three of these lenses:
	\begin{itemize}
		\item Metaphysical: The world is composed of things in systems. These things and systems follow laws. 
		\item Ethical: Technology can be used to kill people.
		\item Epistemological: There is an external world tat can be known. Everything in this external world can be improved upon if we want to. 
	\end{itemize}

	Technology and philosophy are not opposites, they work well together and influence each other. 
	
	Bunge separates technology and science. Technology aims to find the truth for useful problems, while science aims to find the truth for the sake of science. Science is pure and neutral. It just aims to find the truth whether good or bad. Technology introduces values so it could be good or bad. Think of nuclear bombs, the science is the nuclear reactions. This is neither good or bad. But the technology is the actual taking the science and creating a bomb to kill people which is probably a bad thing. 

	There is a philosophical input and output to technology. We can see that culture and society affects technology, but conversely technology also affects culture and society. 

	Bunge argues that philosophy of technology is a subdicipline of philosophy of science. 
	
	
	\subsection{Vienna Circle}

	These people came up with the theory of logical positivism which basically rejected everything except for purely scientific methods. Everything else was meaningless. They thought that metaphysics uses too much language, and logic is the only valid part. This contains concepts, arguments, and propositions. 

	Scientific realism means that an object exists outside of the mind, propositions must be true or false, and arguments must be valid. 

	For a statement, it can either be A Priory (Analytic) or A Posteriori (Synthetic). A Priori means we can know without any experience, just based on prior knowledge such as “Tomorrow it will rain or not rain”. A Posterior means we need to experience in order to know if it is true or not such as “Tomorrow it will rain”. We don’t know if that is true or not without waiting till tomorrow, going outside, and seeing if it is raining or not. Logical positism groups this into two boxes. A posteriory synthetic, and A priory analytic. Other philosophers seperate synthetic and a posteriory and A priory and analytic. The difference is the A posteriori and A priory are epistemological distinctions, while analytic and synthetic are semantic distinctions. 

	The problems with this theory is that if everything needs to be verified scientifically, then how do we verify this theory itself? How do we scientifically verify the statement of “We need to scientifically verify everything”? Also, how do we verify other statements such as “Murder is wrong”? We know that is a correct statement intuitively, but we cannot verify it by logic arguments, and it is not something we can see in the world. 
	
	
	\subsection{Paradigm Shift - Thomas Kuhn}
	Thomas Kuhn groups science into different paradigms. These paradigms are basically like environments with a certain set of assumptions, rules, problems, and solutions. 

	Over time, people will be doing normal science. This is where scientists have some problems that need to be solved and they solve these problems within the current paradigm using the current rules and assumptions. This works quite well until we get to a problem that cannot be solved in the current paradigm (with the current rules, and perspective). This is called an anomaly, which is an inconsistency in the paradigm. Often we ignore or come up with an explanation for the first few anomalies until they get too big to be ignored. Then once the problems are too big to be ignored, and start to prevent solving multiple scientific problems, we enter a crisis in the paradigm. 

	Then once we are in a crisis, we will have a scientific revolution which will replace the old paradigm with a new and improved paradigm. This can have new rules, new assumptions, and new ways of thinking. Then we are back into the normal science but now in this new paradigm (environment).

	This cycle continues on until the next crisis.
	
	
	\subsection{Martin Heidegger}
	A device is called available if we use the device, and do not think about us using the device, such as a hammer, or a car. These devices are meaningless. 
	
	Technology causes people to lose all meaning since we use it for everything. 
	
	Technology is a mode of revealing. For example, we have energy in the air and the windmill reveals that energy. Something like art is not actually created, it has always existed, it was just existing hidden in the mind. Technology allows that art to be revealed so we can see it. 
	
	In modern times, we do not exactly use technology to naturally reveal stuff. We challenge forth the hidden things from nature. We think as nature as a standing reserve of energy, and that it is there to be used, and challenged. For example, in previous times we would simply burn fuel to create heat. Now we challenge this fuel to get the most energy possible from it to run an ICE. Even the act of getting in the future now involves complex technology to extract oil from the ground, or to efficiently chop down forests when talking about wood. 

	Modern technology wants to categorize into boxes that can be controlled and challenge nature which causes individualism to be removed. 
	
	
	\subsection{Albert Borgmann}
	Borgmann is a middle ground between technology determinism and technology as an instrument.

	He believes that technology alienates humans from reality. He thinks of technology as being paradigm based. It is a framework to do science. One paradigm is premodern technology, and another one is modern technology. He does not care about what technology is, only its impact on society. 

	We cannot live without technology, but we can pick focal devices which allow us to have meaning in our lives such as running. We can still use technology in our lives, but we can focus on running. 

	In the past we had to chop wood, and stoke the fire. This gave us meaning and a relationship with nature, and other people to organize all of this. Now we are spoiled with automatic systems, so we don’t have to worry about this. We are disengaged with reality. 

	People are consumers because we are detached from reality.

	People can be couch potatoes since they don’t really have to engage with reality `at all.
	
	\subsection{Autonomy of Technology (Jacques Ellul)}
	The real question here is if technology posesses a will of its own. 
	
	He makes the distinction between hard determinism, where technology determines \textbf{everything} in human life (Ellul argues this) and soft determinism where technology influences everything, but there are other factors that determine human life (heilbroner argues this).
	
	Ellul argues that technology is the prime mover of human life and it changes peoples notions and hopes for everything in life. 
	
	He argues that we do not understand how this technology operates, so we cannot truly cannot make informed decisions without understanding it. 
	
	Technology can continue for a long time even without science, since technology dominates science. Technology has seperated itself from science and it not neutral, it has morals.
	
	
	\subsection{Robert Heilbroner}
	Heilbroner argues for soft determinism. He argues that technology is a guiding force in evolution, but we still have freedom. 
	
	Heilbroner argues that technology is sequential, this is true since medieval could not possibly have had nuclear technology. 
	
	This sequential pattern of technology means that there is definitely a fixed sequence to technology, and it moves at an incremental pace and is fairly predictable. This can be shown since often different scientists will come up with the same concepts at the same time, completely independently such as calculus. 
	
	Heilbroner argues soft determinism because it is the humans that ultimately make the discoveries, and they are the labour force. He believes that technology is not the prime mover, but it is a mediating factor. 
	
	\subsection{Moral Theories}
	We have four main ethical theories:
	\begin{enumerate}
		\item Utilitarianism
		\item Deontology
		\item Virtue Ethics
		\item Feminist Ethics
	\end{enumerate}
	Utilitarianism means that the outcome of an action is what makes it right or wrong. We want to maximize pleasure, and minimize pain. We can quantify the amount of pleasure/pain. 
	
	Deontology means that the intentions matter in a decision. Killing someone is wrong regardless of the result. All good decisions should be universally considered good. It says that humans should be treated as people, not just as a means to get something. 
	
	Virtue ethics say that ethics is not just about acting, but it is about living and thinking about who should I become? It is about trying to do the right thing, and have the good life. It says that we need to have not too much of a feeling, but also not too little of a feeling (Not cowardice, but not foolheartedness either).
	
	Feminist ethics means that female morals are different than male morals. Male morals focus mostly on logic, while female morals focus more on relationships and compassion. 
	
	\subsection{Deep Ecology}
	Shallow ecology is about using technology to come up with a fix to climate and ecology problems such as recycling, organic farming, and electric cars. It considers nature important only because it is a resource source that we need to preserve so humans don't die out. 
	
	Deep ecology agrees with shallow ecology, but does not believe it is robust enough. It believes that we need to go deeper and change our whole ways. We need to completely change the current social, economic, cultural, etc, paradigm. There are some issues with the current social paradigm since it is the source of all problems (consumption). Currently new is more important than old, and we have a very high standard of living. These ideas need to be changed. Deep ecology argues against individualism, and thinks in a more whole manner. 
	
	\subsection{Deeper than Deep Ecology - Feminist Ecology}
	This basically says that the center of deep ecology is rational thinking, and that we ignore female thinking. It argues that females are more connected to nature since they are the source of life, and therefore male thinking cannot fully realize deep ecology. 
	
	\subsection{Nuclear Power - Ethics}
	The main question here is if we have an obligation to preserve future generations. Nuclear power does have its problems, especially in the future with nuclear waste. The question is that do we need to deal with that now in the present or can we just bury it in a hole and leave it for future generations to deal with. 
	
	One point is that future generations do not exist so how do they have rights? Also, by convention we usually punish offenders, but how will future generations punish us? Against this, we can state the concept of an unacquired contract which is a contract that is assumed, and not written down. There may be an unofficial contract between us and future generations. 
	
	Some people argue that we have more of an obligation to the present than to the future since in the future there will be more technology, and they will be better equipped to deal with this problem. 
	
	An argument against nuclear power is that we morally cannot shift the costs of a service onto a completely unrelated group of people. 
	
	This concludes by saying that we do have obligations to the future, and while nothing is certain in the future when it comes to nuclear power, we have probabilities off which we can act. 
	
	\newpage
	\section{Midterm 1 Practice Questions}
	
	\subsection{Explain the differences between how the Greeks viewed technology, vs how modern society view technology.}
	
	Greeks viewed technology as complementing nature. Now we see technology as opposing nature. In the time of the greeks, they thought as nature as divine. Nature was this very important thing that we all should learn to imitate. Technology was seen as something that would complement and improve nature. The Greeks thought of technology as imitating nature such as how a house would imitate how animals in nature build nests, and traps would imitate spider webs. Their line of thinking with a house would be that the base idea is a bird nest which is found in nature. Then the humans would improve it by adding a roof, adding a door, and other things to make it better. 
	Nowadays we think of technology as opposing nature. We think of nature as things such as birds, lions, trees, and grass. We think of technology such as farming, and building as destroying animals habitats, and destroying trees and land. We also think of technology such as computers using lots of electricity which also destroys nature.
	
	\subsection{Come up with a definition of technology.}
	Technology is anything that is created by humans, or something that has already been created by humans, for the purpose of doing something, making something easier, or entertainment. 
	
	This definition includes many things that we know to be technology such as computers since computers exist for all three of my points. They “do something”, make life easier, and can be used for entertainment. Saddles for horses were invented by humans for the purpose of more easily riding horses. Hammers were invented to more easily apply forces to objects. Saws were invented for more easily cutting certain materials. Even Art is classified as technology through this definition since Art is for entertainment. It also includes items made by items made by humans, such as an art piece made by Artificial Intelligence, or a binary program file compiled by a computer. Something like a water bottle, or a plate are also classified as technology since they were both invented by humans in order to make it easier to satisfy the basic human need of food and water. 
	
	Many things we know not to be technology are not included in this definition such as cows, trees, and the ground. This is because none of these were created by humans or by something made by humans. These all existed for a long time and would still have existed even without humans ever existing. If I was to go into a forest where no human had ever been, we can all agree that that is nature. My definition also says that is nature since none of that was created by humans. 
	
	\subsection{Is there any inconsistency with Bacon's “Subduing nature by shaking it to its bones”?}
	Bacon mentions that we need to obey nature’s laws in order to subdue it. What he means by that is that in order to conquer and overcome nature, we need to first know how it works. For example, in order to cut down a tree to get the wood, which can be used to build houses, tools, or many other items, we first need to understand the physics. If we do not know that, we will cut down a tree and it will fall in any direction, potentially on the person cutting it down. Another more modern example is computers. To make the computers, which overcomes nature in so many ways, we first needed to understand physics. At the base level, computers are just physics. The directions and behaviour of electrons define how a MOSFET transistor works. Transistors are what the logic gates in a computer are made of, and these logic gates build the whole computer. 
	
	Even once we have the computer built, we still need to know how to follow logic laws to make the programs that run on these computers. 
	
	Another thing Bacon mentions is the theory of the idols. Since there are naturally all the 4 idols such as the idol of the cave, idol of the tribe, idol of the marketplace, and idol of the theater, these are all things that we must understand in order to conquer them. The first step in overcoming a preexisting bias is to understand that it exists. Here we are understanding that these four idols (biasses) exist in the human mind by nature, and we can therefore conquer them and reduce their effect on our scientific research. 
	
	In conclusion there is no real inconsistency in Bacon’s statement of having to understand and obey nature in order to conquer it and “shake it to its bones”.
	
	\subsection{Write about the good and bad effects of technology on humans. How does technology improve things, and how does technology erode human values?}
	Technology has a large impact on humans, and this impact has both positive and negative impacts. The main impact of technology is to improve efficiency. We see this all the time in real life. When we need to get somewhere, we can drive, or take the bus, or even fly. This means we are able to meet people from all over the world. If this technology did not exist, people would have never been able to discover other nations. Canada would be full of people whose ancestry is Canadian, Europe would be full of people whose ancestry is European, and so on. Now we have people from all over the world in Canada and other countries around the world. It also increases efficiency with manufacturing allowing us to have many luxuries improving quality of life such as computers, smartphones, and so on. 

	Technology also provides basic necessities to the vast majority of people such as food, water, and shelter. In the past, people would make a shelter out of sticks and leaves which would get destroyed during a storm. It would have taken an enormous amount of effort from a large number of individuals to make any large sturdy structure. (Building a large structure could be argued is technology as well). It would have taken such effort that only a small percentage of the population would have been able to get a sturdy shelter. Now we can build a strong sturdy building in as little as a few days by a single person if needed depending on the size and complexity of the building (considering a small shed, this can be done by one person in a few days). It also allows everyone to have food, and especially the correct types of food year long. In the past, it would have been challenging to get vegetables and fruits in the colder months due to them not growing. Now we can easily get them due to shipping, and fridges. Even water is much easier to get as we can pump it from a water source all the way across the country to people’s houses. 

	Despite all of technologies benefits with efficiency, we have to think of the hidden costs of this efficiency. This is actually quite visible especially over the past few years with the rise of generative AI. One example is if we purchase a product and it is defective, and we require support, we often have to go through an AI bot to get support. While a good human would likely realize the mistake and provide a replacement product as should be done, the AI bot may not as it has different values than humans. Humans are compassionate beings, and AI is just made to follow instructions. This shows that technology erodes human values such as compassion, and justice. Another example is that even with driving, we often spend so much time driving around, which is more efficient, but is that really what we want to do in our free time? 

	One huge point is that these technologies erode our freedom. Nowadays, we cannot practically drive to a location in the middle of nowhere, and camp for a month. If we do this, we will miss bills due dates, potentially miss important letters in the mail such as jury summons, as well as jobs would not likely allow this. This shows that technology really erodes the sense of freedom that humans have. We are bound by this technology so we may not be able to do what we truly want to do. 
	
	Another very recent example is modern safety equipment on cars. In the past, we could take a car and drive wherever we want, do whatever maneuvers we want, and the car would follow our instructions. Now the car has many advanced safety features that are meant to improve safety using technology, but as a side effect, the car no longer always will follow the instructions of the driver. For example, if I press the gas pedal, and the car senses an item in front of the car, it will press the brakes without me doing so. Regardless of the intentions of the car, this has reduced the freedom of the user. Then are we really the ones driving the car or is it a machine doing most of the hard work? An analogy is where young people go shopping with their parents and then are proud of going shopping. At the end of the day, the parents were still the ones making all the important decisions. They were just trying to make the child feel good about them being supposedly independent. Is this is what is going on with cars where the car is trying to make the driver feel in control when all the strings are really being pulled by the machine?

	These technologies also erode peoples creativity. In the past before technology, people came up with ideas that were truly their own. Now, often people see something online, or talk about something with someone from a different part of town, and then come up with an idea. Is this idea truly their own? Or is the idea just copied from what they saw online, or from this other person from a different part of town?
	
	
	\newpage
	\section{Midterm 2 Practice Questions}
	\subsection{What is the difference between synthetic apriority and analytic apriority?}
	An analytic statement means by definition it can be found to be true. So for example all bachelors are unmarried is by definition. Synthetic means just the way the world is. So this is something like every event has a cause. The way the world is, every event has a cause. A priori means we know it without experience and observation, just by prior knowledge. So both above statements are a priori since we don’t have to test in order to confirm that it is true, we know it is true before observing it in real life. 
	
	\subsection{What is the debate between logical positivism and Immanuel Kant?}
	This debates whether or not a statement can be both synthetic and a priori. Kant says that a statement can be either analytic a priori, synthetic a posteriori, or some statements can be synthetic a priori. This means the statement can be known without observation, but it is not by definition. An example is the statement 7+4=11. The definition of 7+4 is not 11. But we can logically reason that it is 11 without taking 7 apples, and 4 apples, and counting 11. If we are however using logical positism, this would end up likely being classified as analytic a priori. 
	
	\subsection{What is the process of the paradigm shift?}
	We have paradigms, this is an environment with its own set of rules, and assumptions for science. This is very similar to the aspect of containerization used in computers. For example, I host some websites and services myself. I tend to seperate these into containers that each have their own operating system with their own dependancies, paths, and access to their own resources. Paradigms are similar in that they have their own rules and assumptions for science. The process of paradigm shift is where we start with normal science which has a paradigm called A. We have rules and assumtions associated with A. Scientific problems appear and we solve them using the rules and assumptions of A. Then an anomaly appears, this is a problem that cannot be solved in this environment A. We often will either explain or ignore this anomaly. This goes on until there are too many anomalies to ignore or explain away. Now we have a real problem on our hands, the current paaradigm rules are not enough to solve the problems. So we enter a crisis. Here we change the viewpoint and rules to new ones. Now we are in a new paradigm with new rules and new assumptions that can solve the previous anomalies. This then becomes the new normal science and continues until we find a bunch more anomalies, then get a crisis, and then have another scientific revolution to get to a new paradigm, and repeat again. 
	
	\subsection{What are the two objections of the principle of verification?}
	The principle of verification means that every statement must be scientifically verified in order to have meaning. The two objections are firstly, there are some statements that we know have meaning but they cannot be verified. For example, how do we verify the statement that murder is wrong? Or theft is wrong? We cannot really verdify that something is “wrong”. We can verify that people are sad after a murder, but that does not necessarily make it wrong. We are sad at a funeral but that does not make a funeral wrong. The other main objection is that if all statement need to be verified scientifically, how do we verify that claim itself? Meaning, how do we verify the claim that “All claims need to be verified or the claim is meaningless”? If we cannot verify that claim (which we cannot), then that claim is meaningless which breaks the whole thing. 
	
	\subsection{What is borgmanns views?}
	Borgmann thinks that technology is everywhere and we cannot really escape it. He thinks that technology takes away our connection with reality. In the past, we had to do a lot of work. Now we trade this work for consumption. So when we are driving, we are consuming fuel. We are not actually doing anything in real life. There is a disconnect between the real world and what we are currently doing. He thinks that we need focus things which are things that connect us to reality such as running. Here we are doing physical work, but we are also getting something done. 
	
	
	\subsection{What does heidegger say about premodern vs modern technology?}
	Heidegger thinks that modern technology is not good. Modern technology takes away the meaning of life since we do not think about all the devices that we use in our day to day life. We think of these devices and technology as just a standing reserve of resources whether that is thinking of trees as just a heat source (to burn), animals as just a food source (to eat), or lakes as just a sustenance source (to drink and clean). We do not actually think about anything which would cause us to lose touch with reality. We try to think of everything as just a resource and remove all the meaning from items, and just categorize them as some sort of resource. We challenge forth the resources from a thing. For example, take a cow. We raise this cow in an enclosure, then put it in a barn to eat food until it gets big and fat. Then we kill the cow and get the meat. We are shaping this cow, challenging it, to become a resource for us to eat. Heidegger thinks that premodern technology was simply a form of revealing the true nature of an item. When we burn wood for a fire, we reveal that it is actually heat. When we grow a cow, we think of the cow as always having been there, we just raise it to grow up. Modern technology is similar to premodern technology, but taken to an extreme to only view these items as a standing reserve of resources. 
	
	\newpage
	\section{Final Exam Practice Questions}
	
	\subsection{Come up with a solution to the climate crisis}
	
	My thinking is that as deep ecology mentions, the best way to solve this crisis is to completely change the way society is. Reduce consumption, reduce the need for travel by car, and just overhaul the entire way of living. As I mentioned, this would mean we could no longer get a new cell phone every year, we could not commute across town to get to work every day, we would probably have to walk. We should not transport tons of stuff by plane across the globe, and so on. Personally, while I agree with the deep ecology people and these concepts, I don't think that this will ever happen practically. And when thinking of it from an ethical perspective, it might not be justified. Using deolontoligical ethics, where the intent matters, it is fine, there is no problems. We are trying to save the planet to help humans in the long term. However, according to the utilitarian ethics where we want to maximise happiness, we would definitely not be maximising hapiness since people would be quite sad about losing their phones, and cars. This is definitely true since now many people cannot survive even a day without their phones, and other technology without panicking and getting quite upset. 
	
	One assumption that this solution makes is that the whole world would be on board with this solution. Almost every solution to a climate crisis would have a simple loophole of people/companies just moving to a different country where these rules are non-existant. And this would be the case unless the whole world (or the majority at least) have similar rules. 
	
	My solution is based on two major concepts. The first is how corporations are defined. Corporations are legal entities that have shareholders. These shareholders have one purpose which is to make money. If the corporation is not making money, then there is a big problem. 
	
	The second is based on shallow ecology concepts of coming up with a technological "fix" for these problems. As mentioned before, I think that while deep ecology would be better, as a whole society I do not think we would all be on board with that. 
	
	In many fields, there exists better technology than what is on the market, but it is not popular due to cost/complexity problems. I am familiar with computers, and I know the difference between an asynchronous processor, and synchronous processor. Asynchronous processors are significantly more efficient (less power usage) which would obviously help the environment massively. But they are not really researched or explored due to the massive complexity over the current synchronous designs. There are examples like this in every single product field that exists. There are better solutions that are just not explored due to cost constraints. 
	
	My suggestion, is to enact government regulation on these corporations to severely limit their emmisions and environmental impact of their products. These regulations would have to be strict enough to force all these companies to pursue cutting edge research on new ways to solve their product, since there would be new rules. These companies would have a lot of motivation to do this, due to them being a corporation, and they must make profit for the shareholders. These companies employ millions of smart engineers, who could definitely come up with some quite amazing solutions that not only benefit the envoronment, but are still enjoyed by consumers (in order to make profit). 
	
	My suggestion is not to force society as a whole to change our habits, but to force corporations to change their habits (basically go into a new paradigm) which would ideally not directly impact humans. For example, if we require Apple to cut power usage of their iphones 10x, and they release a new iPhone that is 10x slower, then consumers just won't buy it, and stick with the old ones. That is unacceptable for the Apple shareholders, so Apple would be forced to come up with a new idea that satisfies the regulations. Apple would have the resources to do this as they are a huge company with lots of engineers to think of and design better solutions. 
	
	This solution that I have come up with may not be enough, but I think it would greatly help, and it is actually practical, and does not violate moral principles of current citizens.
	
	\subsection{Come up with definitions for the following concepts}
	
	\subsubsection{Techne}
	Greeks thought of Techne being mechanical knowledge such as how to use an axe, or how to drive a car. It also includes art such as how to paint. 
	
	\subsubsection{Episteme}
	Greeks thought of Epistems as the theoretical knowledge of techne. So basically how techne works. So it would be the physics of how an axe works, or how the mechanics of a car work. 
	
	\subsubsection{Physis}
	Physis is basically what greeks though of nature. It is the process of something growing up and coming to be. So plants would be considered nature since they start small, and grow into a tree. A rock also be considered physis since while it does not grow, it is not made with any human intervention, and it has its own growing process (which for the rock is no growth). 
	
	\subsubsection{Four Idols (Biasses) of the human mind}
	There are four idols suggested by Bacon. These are things that can influence the human mind to have biasses. The first is idol of the cave. These are thoughts that everyone thinks in the same way as me. So I could think that everyone thinks the same way as I do and I am influenced by the cave. The second one is the tribe which are biasses existing just by human nature such as wanting order, and emotions. Idols of the marketplace are about misunderstandings and language barriers between people causing problems. Finally, idol of the theater is about superstitions that can cloud judgement. 
	
	\subsubsection{Three stages of human society development (Compte)}
	Over time, there are three main stages of humen society as a whole. The first is the theological era, this is when society is mostly governed by the military and religion. Most things are not explained well, they are explained as supernatural things or by gods. Then there is the metaphysical era which is when society is mostly governed by lawyers, and religion. Many things start to be explained using vague and lofty terms like nature. Then finally there is the scientific era where society is generally governed by science and reason. This is where most things are explained through rigorous science and observations. 
	
	\subsubsection{Positive Philosophy}
	Positive Philosophy states that knowledge must be gained through observation, experimentation, and verification. Everything must be based on logic. 
	
	\subsubsection{Sociology}
	Sociology is a very broad science that is meant to fix the problem of humans being too fixated on small elements of science, while ignoring the bigger picture. It is a whole unified science that is the center of positive philosophy.
	
	\subsubsection{Technological Determinism}
	This is the fact that technology is deterministic, it has a predetermined pattern. In the same situation, with the same inputs and environment, it will always give the same output. 
	
	\subsubsection{Three branches of philosophy}
	There are three main branches of philosophy, these are the ethical, epistemological, and metaphysical. 
	
	\subsubsection{Logical Positism}
	This is about eliminating any statements that are known through prior knowledge but not by definition of the word (synthetic a priori) and saying that everything must either be known by definition, or observed through science. It is also about eliminating metaphysics, and saying that for any statement to be meaningful, it must be verified by experience, or logic/math. 
	
	\subsubsection{Principle of verification}
	This is about saying that any statement must be verified either through experience, or by logic/math. Otherwise the statement is meaningless. 
	
	\subsubsection{Enframing}
	This is about seeing the world as simply a bunch of resources to be used efficiently. We treat everything as just a standing reserve. 
	
	\subsubsection{Standing Reserve}
	This is about treating something simply as a reserve of resources, just sitting there to be used. For example, we think of trees as just lumber waiting to be chopped down and used. 
	
	\subsubsection{Bring Forth vs Challenge Forth}
	In the past we would bring forth the energy from resources, such as wood and gas. Now we challenge forth to get lots of energy from these resources. Think of motion. In the past we would walk around, or maybe ride a horse. We would cooperate with nature to do what we want. This is not really changing nature much, or shaping it to work for us. Now, we take things such as gas, put it in this complex machine and explode it to move around in a car. Now we force nature to do what we want in the way we want to maximize our happiness. 
	
	\subsubsection{Focus Device}
	This is something that we can focus on in our lives that gives us a connection to nature, like running. This is useful since it combines the current effort with the current reward that is missing in modern life. 
	
	\subsubsection{Device Paradigm}
	This is about having devices everywhere to make life simpler. These devices are instand, easy, and available. While this is good in theory, it means we no longer have to think about anything really, so we lose a social experience, and an opportunity to connect with nature. One example is the hearth and fireplace in past times. 
	
	\subsubsection{Soft Determinism and Hard Determinism}
	Determinism in this sense is where technology is the driving force behind civilization. Hard determinism means that humans have no freedom, it is completely determined by technology. Soft determinism is the same idea that humans are guided by technology, but it is different since it mentions that humans are the ones actually doing the labour, so humans do still have freedom. 
	
	\subsubsection{Four Main Moral Theories}
	The four main moral theories are Deontology (the intentions matter), Utilitarianism (the outcome matters), Virtue ethics (where we do morally good actions that leads us to be a good person), and feminist ethics (females tend to think more about relationships, males tend to think more of pure logic when it comes to difficult situations).
	
	\subsubsection{Deep vs Shallow Ecology}
	Shallow ecology is about coming up with technical solutions to the climate crisis. The key point is that it is coming up with solutions since the resources are depleting, and humans need them. Deep ecology says that shallow ecology is good, but it is not enough. We need to basically change the entire social paradigm to move away from a consumption based society (that wants the latest everything) and change the entire structure of society. The difference is that deep ecology says we need to protect the environment for more reasons than just it affects humans. 

	
	
	
\end{document}