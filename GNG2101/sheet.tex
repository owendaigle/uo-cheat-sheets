% Crucial Preamble
\documentclass[12pt,letterpaper]{article} \usepackage{amsmath} \usepackage{graphicx} \usepackage[margin=1in]{geometry} \usepackage{longtable}  \usepackage{amssymb}

% Extra Preamble
\usepackage{fancyhdr} \usepackage{enumitem} \usepackage{float} \usepackage{soul}
\usepackage{multicol} \usepackage[compact]{titlesec}
\usepackage[section]{placeins}
\usepackage{pdfpages}

% frames with display breaks
\usepackage{mdframed}
\allowdisplaybreaks

% change spacing
\usepackage{setspace}
\setlength{\parskip}{0.4\baselineskip}

% Remove paragraph indentation
\setlength{\parindent}{0pt}

% Reduce space before and after section headings
%\titlespacing*{\section}{0pt}{0.1\baselineskip}{0.2\baselineskip}

% changes font
%\renewcommand{\familydefault}{\sfdefault}

% adds header and footer
\pagestyle{fancy}
\fancyhead{} \fancyhead[C]{GNG 2101 Summary Sheet} \fancyhead[L]{GNG2101} \fancyhead[R]{Owen Daigle}
\fancyfoot{} \fancyfoot[C]{\thepage}


\begin{document}
	
	\begin{center}
		\Large\textbf{GNG 2101 Summary Sheet} \\
		\vspace{0.5em}
	\end{center}	
	\tableofcontents
	\newpage

	\section{Design For X}
	Functional requirements define what a product should do. DFX (Design for $X$) allows us to ensure the product meets the certain requirements called $X$. 
	
	Non functional requirements are about How the product does what we want it do do such as the speed, and quality. 
	
	DFX allows us to focus on specific goals for the product, one at a time. We want to have discussions about "How do we incorporate $X$?". Each $X$ has specific design standards and rules that help to incorporate this $X$. 
	
	Some examples are:
	\begin{itemize}
		\item Design for Accessibility
		\item Design for Reliability
		\item Design for Testability
		\item Design for Repairability
		\item Design for Compliance
		\item Design for Sustainability
		\item Design for Maintainability
	\end{itemize}
	
	\subsection{Compliance}
	Compliance constraints are stuff such as health and safety regulations, environmental regulations, quality codes, and standards.
	
	Often devices need to be able to use generic designs such as USB which means they need to comply with these preexisting standards. These standards are usually explained in technical documents. 
	
	Government regulations for environment, and health/safety are also critical to avoid fines. 
	
	\subsection{Sustainability}
	\subsubsection{Triple Bottom Line}
	Sustainability is a concept that depends on people, profit, and the planet. We need a product to satisfy all three of these concepts for it to be able to be sustainable which is challenging. So we need a balance between all three of these items. 
	\begin{figure}[H]
		\centering
		\includegraphics[width=0.4\linewidth]{"images/triple bottom line"}
		\caption{Triple Bottom Line Graphic}
		\label{fig:triple-bottom-line}
	\end{figure}
	
	
	\subsubsection{Life Cycle Assessment}
	This is a way to determine the impact of a certain device over its life cycle from when it is created in the factory (or before), to when it goes into the landfill (or potentially after). We analyze the inputs, outputs, and potential environmental impacts of the product over its lifetime. 
	
	We start by determining the frame of the LCA analysis, we can either do:
	\begin{itemize}
		\item Cradle to Grave - From raw materials to end of life
		\item Cradle to Gate - From raw materials to gate of manufacturing facility (before transportation to consumer)
		\item Cradle to Cradle - From raw materials, to end of life, and then it is recycled into new raw materials
	\end{itemize}
	
	We have a bunch of stages in the LCA analysis such as:
	\begin{itemize}
		\item Raw materials acquisition
		\item Materials manufacture
		\item Product manufacture
		\item Product use / Consumption
		\item Disposing of product (recycling, or dump)
	\end{itemize}
	
	\subsubsection{Normalization}
	This is a way to calculate greenhouse gas emissions using a standard metric. This allows us to fairly compare products to other products to see which products create the least emissions. 
	
	\subsection{Manufacturability}
	This is about designing products that can efficiently (and actually) be manufactured. We have to know actually how to make each part of the product, such as what machine will make the part, and how it will be made. 
	
	We need to make some sort of sketch of the product and its parts with dimensions such as a CAD model.
	
	A few points to keep in mind when designing for manufacturability are:
	\begin{itemize}
		\item Avoid customization to make the process easier, use already existing products if possible.
		\item Reduce the number of components as much as possible
		\item Use processes that are faster for prototyping
		\item Select the correct tool for the job
	\end{itemize}
	
	Below is a bunch of tools with what they \textbf{should}, and \textbf{should not} be used for:
	
	\begin{table}[H]
		\centering
		\caption{3D Printer}
		\begin{tabular}{p{0.45\textwidth} | p{0.45\textwidth}}
			\textbf{YES} & \textbf{NO} \\
			\hline
			\begin{itemize}
				\item Complex geometries
				\item Minimal Supports
				\item Small Parts
				\item Brackets and Adapters
			\end{itemize}
			&
			\begin{itemize}
				\item Flat Items
				\item Threads
				\item Weight Bearing parts
				\item Waterproof parts
			\end{itemize}
		\end{tabular}
		
	\end{table}
	
	\begin{table}[H]
		\centering
		\caption{Laser Cutter}
		\begin{tabular}{p{0.45\textwidth} | p{0.45\textwidth}}
			\textbf{YES} & \textbf{NO} \\
			\hline
			\begin{itemize}
				\item Flat cuts
				\item Enclosures
				\item Templates
			\end{itemize}
			&
			\begin{itemize}
				\item Round Materials
				\item Structural Parts
				\item Parts that need to be screwed into
				\item Thick materials
			\end{itemize}
		\end{tabular}
		
	\end{table}
	
	\begin{table}[H]
		\centering
		\caption{Arduino}
		\begin{tabular}{p{0.45\textwidth} | p{0.45\textwidth}}
			\textbf{YES} & \textbf{NO} \\
			\hline
			\begin{itemize}
				\item Low power low voltage circuits
			\end{itemize}
			&
			\begin{itemize}
				\item High power circuits
				\item Circuits that don't need code, only power
			\end{itemize}
		\end{tabular}
		
	\end{table}
	
	\begin{table}[H]
		\centering
		\caption{Drill Press}
		\begin{tabular}{p{0.45\textwidth} | p{0.45\textwidth}}
			\textbf{YES} & \textbf{NO} \\
			\hline
			\begin{itemize}
				\item Circular Straight holes
			\end{itemize}
			&
			\begin{itemize}
				\item Precise hole position
				\item Not all the way through a hole
				\item Large holes, or Square holes
			\end{itemize}
		\end{tabular}
		
	\end{table}
	
	\begin{table}[H]
		\centering
		\caption{Mill}
		\begin{tabular}{p{0.45\textwidth} | p{0.45\textwidth}}
			\textbf{YES} & \textbf{NO} \\
			\hline
			\begin{itemize}
				\item Metals and Plastics
				\item Slots and Holes
				\item Holes not all the way through
				\item Flat sides
				\item Precise Machining 
			\end{itemize}
			&
			\begin{itemize}
				\item Circular Geometry
				\item Don't use wood
				\item Sharp Inner corners
			\end{itemize}
		\end{tabular}
		
	\end{table}
	
	\begin{table}[H]
		\centering
		\caption{Lathe}
		\begin{tabular}{p{0.45\textwidth} | p{0.45\textwidth}}
			\textbf{YES} & \textbf{NO} \\
			\hline
			\begin{itemize}
				\item Coaxial Geometry
			\end{itemize}
			&
			\begin{itemize}
				\item Square Parts
				\item Long parts
				\item Don't use wood or 3D printed parts
			\end{itemize}
		\end{tabular}
		
	\end{table}
	
	\FloatBarrier
	
	\subsection{Ethics}
	Ethics are the duty of us (the engineer) to society as a whole.
	
	We definitely want to do good work, and maximize benefits and minimize harm. We also want to respect people, and treat them fairly. We do this by using a lot of different ethical lenses. 
	
	These are ways of thinking to guide us to evaluate a decision from different perspectives. 
	
	We have four lenses:
	\begin{enumerate}
		\item Utilitarian
		\item Rights
		\item Common Good
		\item Virtues
		\item Equity
	\end{enumerate}
	\subsubsection{Utilitarian}
	For the utilitarian lens, we need to consider the overall happiness or welfare the action will bring about. 
	
	We could say that minimizing costs of a product would increase overall happiness since more people could benefit from the product. 
	
	We could also say that having good customer service would increase overall happiness.
	
	\subsubsection{Rights}
	For the rights lens, we want to ensure that people's rights are upheld. This includes but is not limited to:
	\begin{itemize}
		\item Right to life
		\item Right to liberty
		\item Right to privacy
	\end{itemize}
	
	We could say that reducing the environmental impact of a product would uphold the rights of future generations to a clean environment.
	
	We could say that having good data protection measures in software upholds the users rights to privacy. 
	
	\subsubsection{Common Good}
	The common good lens is about emphasizing the well being of a community as a whole such as having an accessible health care system, having world peace, and having a just legal system. 
	
	We could say that designing for accessibility would be for the common good since it benefits the community as a whole.
	
	\subsubsection{Virtues}
	A virtue is a character trait that is considered to be morally good. It is a behavior that is right, and avoids being wrong. 
	
	Virtue ethics asks the question of "Who do I want to be? What choices do I want to make?".
	
	We could say that adding accurate and not misleading product descriptions would be good from the virtue lens because it is showing the character trait of honesty. 
	
	We could also say that designing for accessibility is good in the virtue lens since we are showing the character trait of fairness. 
	
	\subsubsection{Equity}
	Equity means that everyone gets their due. It does not mean that everyone is equal, but everyone gets what they need to be successful. For example, someone who is poor might get a lot of government assistance and food, while someone who is rich would not get any of that extra help since they are already doing well. 
	
	The equity lens is about ensuring the appropriate distribution of benefits and burdens. 
	
	We could say that adding a discount for low income families would be good in the equity lens since it helps give a better distribution of benefits. 
	
	We could also say that fair labour practices are good in the equity lens since it gives the correct and fair benefits to workers. 
	
	
	\section{Concept Development Process}
	We have a few different design processes such as:
	\begin{itemize}
		\item Waterfall - Sequential process
		\item Agile - Iteratively create a prototype, get feedback, then find solution, and repeat
		\item Spiral - Similar to agile method with different prototypes solving iterative problems, but we add risk analysis
		\item Co-Evolution - Represents evolution of problem space, and solution space, the problem and solution evolve together, useful for ill defined problems
		\item Iterative - build $\to$ test $\to$ refine $\to$ repeat
	\end{itemize}
	
	For all of the design methods, to manage the project we have 4 main points:
	\begin{itemize}
		\item Plan - Is the plan well laid out
		\item Processes - Are the processes well defined
		\item People - Do we have a good team
		\item Power - Distribution of authority
	\end{itemize}
	We can use a GANTT chart to manage the project, this breaks up the project up into multiple sub-tasks, with estimated completion times, and dependencies identified. 
	
	For the Iterative Engineering Design Process (IEDP), we first generate solutions, then check if these solutions satisfy the DFX. If so, we prototype and see if the solution works, if it does not, we go back to the start (generate solutions).
	
	When coming up with the solutions, we need to generate a large amount of them. First we remove all concepts that are not feasible (do not meet needs, or are physically impossible). Then we score these solutions using a decision matrix that contains different factors. Finally we test. 
	\begin{table}[H]
		\centering
		\begin{tabular}{|c|c||c|c|}
			\hline
			\textbf{Criteria} & \textbf{Weight} & \textbf{Car A} & \textbf{Car B} \\
			\hline
			Cost & 0.5 & 8 & 3 \\
			\hline
			Reliability & 0.4 & 3 & 10 \\
			\hline
			Features & 0.1 & 4 & 6 \\
			\hline \hline
			Total & 1 & 4+0.4+1.2=5.6 & 1.5+4+0.6=6.1 \\
			\hline
		\end{tabular}
		\caption{Decision Matrix Example}
	\end{table}
	
	\subsection{Bias and Concept Generation}
	We have a lot of cognitive biasses that can affect our decision process. We need to be aware of these biasses.
	\begin{itemize}
		\item Anchoring Bias - Relying on first piece of information
		\item Blind Spot Bias - Failing to recognize bias in yourself
		\item Confirmation bias - Only trusting information that confirms my beliefs
		\item Negativity bias - Only focusing on negative events
		\item Outcome bias - Only judging based on outcome
	\end{itemize}
	We can try to overcome this bias by using concept generation (creative thinking) strategies.
	\begin{itemize}
		\item Sketching 
		\item Lateral Thinking - Thinking outside the box
			\subitem How would we deal with this in a different time period?
			\subitem How would we deal with it in a different country?
			\subitem How would we deal with it as a different gender or age?
			\subitem How would we deal with it as someone else?
		\item Evolution - Find and review existing products
		\item Synthesis - Combine two products into one
		\item SCAMPER - Substitute Combine Adapt Modify Put to another use Eliminate Reverse
		\item Morphological Analysis - 
			\subitem Identify Problem and Separate Variables
			\subitem List many variations for each variable
			\subitem Randomly select lots of variations
			\subitem Find a good variation to make a whole
	\end{itemize}
	
	\section{Prototyping, Testing, and Iteration}
	A prototype is an early model of a product built to test a concept or process in order to learn something for the final product. 
	
	Since the prototype tests \textbf{a} concept, we almost always need more than one prototype to test multiple concepts. 
	
	We have different levels of prototype fidelity, we can have a low quality low fidelity prototype, medium fidelity, or high fidelity. We also need to come up with a plan to test this prototype. 
	
	When coming up with a prototype plan, we need:
	\begin{itemize}
		\item Prototype Purpose
		\item Prototype Fidelity
		\item Built Plan
		\item Test Plan
	\end{itemize}
	
	Note that these prototypes and tests do not have to be physical, they can be done in software and/or using calculations.
	
	\section{Teamwork}
	Productivity is maximized when the task requirements, team dynamics, and individual situations align well. In practice, this is very hard to achieve. 
	
	Throughout a project, a team has a bunch of stages that it will go through:
	\begin{enumerate}
		\item Forming - Start of team, introductions
		\item Storming - Team conflict and problems
		\item Norming - Reconciling the conflict
		\item Performing - Doing the project well
		\item Adjourning - Wrapping up the project
	\end{enumerate}
	
	To determine how well a team is working together, we use the CARE model:
	\begin{itemize}
		\item Communicate - How well the team shares information, and manages conflict, are roles well defined
		\item Adapt - How well the team manages changing conditions
		\item Relate - How well the team trusts each other to get the work done, and other interpersonal factors
		\item Educate - How well the team learns, reflects, and evolves
	\end{itemize}
	
	One of the major factors we can see here, is how well the team manages conflict. So we need to know of ways that are used to manage this conflict, and come to a solution that is acceptable for everyone on the team. 
	
	Something to note is that not all conflict is bad, some is either good, or will resolve itself. Some conflict will promote better solutions, but it needs to be kept at a reasonable level. However, both sides need to be thinking rationally, not emotionally (want to find the best solution, not want to win).
	\begin{itemize}
		\item Dominating
		\item Integrating (Collaborating)
		\item Accommodating
		\item Avoiding
		\item Compromising
	\end{itemize}
	
	\section{Intellectual Property}
	Intellectual Property is any knowledge or expression created using the brain. We need to know about different ways to protect this IP. 
	
	The way we protect it depends on the medium of the knowledge. 
	
	\begin{table}[H]
		\centering
		\begin{tabular}{|p{0.22\textwidth}||p{0.22\textwidth}|p{0.22\textwidth}|p{0.22\textwidth}|}
			\hline
			& \textbf{Patents} & \textbf{Copyrights} & \textbf{Trademarks} \\
			\hline\hline
			Term & Up to 20 years & 50 or 70 years & 10 years, renewable \\
			\hline
			What is protected & Useful, non-obvious inventions & Art, software, websites, etc. & Designs, shapes, logos, etc. \\
			\hline
		\end{tabular}
	\end{table}
	
	
	\section{Economics}
	
	\subsection{Cost Classifications}
	When running a business, we can classify costs as either \textbf{variable} or \textbf{fixed} (or \textbf{semi variable}). Then we can also classify them as \textbf{direct} or \textbf{indirect}.
	
	Indirect vs Direct is whether it directly relates to a specific project or not. If it does relate to a specific project, then it is a direct cost. Otherwise, it is an indirect cost. 
	
	Variable vs Fixed is whether the cost scales with output or not. For example, the cost of gas for heating a factory is the same whether we are making 1 item a day, or 1000 items per day. However the cost of materials for 1 item is a lot cheaper than 1000 items. 
	
	\subsection{Fundamental Concepts}
	We have two main types of economics. \textbf{Macroeconomics} deals with a national economy, and everything within a country. \textbf{Microeconomics} deals with \textit{one} person, group, or company. We just focus on this one entities economics. 
	
	Money is not static with respect to time, it has a time value. A certain amount of money now is worth less in a year because of \textbf{interest}. We have compound interest, or simple interest. They can be calculated as follows:
	\begin{align*}
		\text{Simple: }(1+i\cdot a)\cdot C_0\\
		\text{Compound: }C_0\cdot\left(1+\frac{i}{n}\right)^{n\times a}
	\end{align*}
	where $C_0$ is the principle amount, $i$ is the annual interest rate, $a$ is the number of years, and $n$ is the number of interest periods/payments per year. 
	
	Due to interest, we can calculate the \textbf{Net Present Value (NPV)} of money at a certain time. This is useful since it allows us to account for time when comparing different income opportunities. For example, getting paid 100\$ now, or 110\$ in 6 months. We can evaluate which one is worth more right now by accounting for the interest we would make on the 100\$ over 6 months. 
	\begin{align*}
		PV = \sum \frac{FV}{(1+\frac{i}{n})^{n\times a}} \qquad \text{OR} \qquad FV = \sum PV(1+i)^n
	\end{align*}
	where $PV$ is the present value, $FV$ is the future value, and the denominator is just the interest calculation.
	
	Depreciation means that new things are worth more than old things. This is especially true with large machines used in business. We use the \textbf{straight line depreciation} equation to calculate the cost of an equipment over time.
	\begin{align*}
		D_L = \frac{\text{Equipment Cost}}{\text{Useful Life}}
	\end{align*}
	
	\subsection{Economic Decision Making}
	
	To make economic decisions, we have a few different ways.
	\begin{itemize}
		\item Break Even Analysis
		\item Sensitivity Analysis
		\item Return on Investment (ROI) and Simple Payback Period
		\item Cost/Benefit Trade-off Analysis
	\end{itemize}
	
	If we are deciding on whether or not to make or buy a certain product, often the cost to make the product will be less for large quantities of the project, but more for just one product. For example, repairing a mobile phone vs paying someone to do so. If I repair a phone, I need to buy a few hundred dollars of equipment and then I can do the repair for cheap. If I pay someone, I might only pay them 100\$ though. If I repair 10 phones, it would be either 1000\$, or if I do it myself, a few hundred in equipment and then small material fees. 
	
	\textbf{Break even analysis} is the point at which the cost to buy equals the cost to make, or the cost to produce equals the revenue. This is a number. Going to the phone example, if we spend 300\$ on tools, and 50\$ per phone if I do it, vs 100\$ if a shop does it, then at 6 phones, if I do it I spend 600\$, vs the shop charges 600\$. We consider the break even point 6 phones. 
	
	\textbf{Sensitivity analysis} allows us to see the profitability of a certain project in different situations based on varying different variables. So we could say "What if we spend 20\% more on development?" or "What if we take an extra week to deploy?" or "What if we sell it for 2\$ cheaper?". We could see the impact on the profit for each of these situations. 
	
	\textbf{Return on Investment (ROI)} is calculated using:
	\begin{align*}
		ROI = \frac{\text{Net Profit}}{\text{Value of Investment}} \times 100\%
	\end{align*}
	Note that we need to account for the time value of money in this calculation. 
	
	The \textbf{simple payback period} is the amount of time it takes to recover the initial investment. This does not take into account the time value of money. 
	\begin{align*}
		n_{tot} = \frac{\text{Value of Investment}}{\text{Net profit per period}}
	\end{align*}
	
	\textbf{Cost/benefit trade off analysis} sows is the advantages and disadvantages on both sides. 
	\begin{align*}
		BCR\text{ (Benefit Cost Ratio)} = \frac{\text{PV of Benefits}}{\text{PV of Costs}}
	\end{align*}
	
	
	\subsection{Financial Statements}
	There are three types of financial statements:
	\begin{itemize}
		\item Balance Sheets (Snapshot of the financial condition at a \textit{certain time})
		\item Cash flow statement (Cash in and Cash out over a \textit{period of time})
		\item Income Statement (Changes in wealth over a \textit{period of time})
	\end{itemize}
	
	For the \textbf{balance sheet}, we just show the equity which is just the assets - liabilities. Assets include current assets, cash, inventory, long term assets, and depreciation (negative asset). The liabilities are accounts payable, and borrowing. The net worth is the equity (assets - liabilities).
	
	The \textbf{cash flow statement} shows the cash in (operations, sales, tax received, borrowing, investments) and the cash out (spending, operation expenditures, bill payments, tax out, asset purchasing, debt paying). We get the net cash flow which is the cash in - cash out.
	
	The \textbf{income statement} (profit/loss statement) shows the sales, and expenses to get the profit/loss. We take the sales, and subtract the cost of goods sold and operating expenses. This is the\textbf{ operating income} (Earnings Before Interest [EBI]). Then once we factor in all interest (a cost to what we are borrowing) we can get the \textbf{Earnings Before Taxes (EBT)}. The \textbf{Net Income} taxes into account income tax.
	
	\section{Appendix}
	
	
\end{document}